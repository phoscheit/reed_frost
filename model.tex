\documentclass[a4paper]{article}

\usepackage[utf8]{inputenc}

\title{A Reed-Frost household model}
\date{}

\begin{document}

\maketitle

\section*{Model description}

We consider a finite population divided into \emph{households}, with individuals
interacting both within and outside of their respective households. We will 
focus on modeling the spread within the household, while taking the possibility
of community-acquired infection into account. 

For a given household with \(m\) individuals, consider at each time \(t\) the
set \(S_t\) of susceptible individuals and the set \(I_t\) of infectious
individuals. The number of such individuals will be noted \(s_t\) and \(i_t\)
respectively. We will assume that infection lasts for a fixed duration \(T_I\),
after which individuals have permanent immunity. The time \(T_I\) is the sum of 
a presymptomatic period \(T_p\) and a symptomatic period \(T_s\). Some 
percentage of individuals experience \emph{asymptomatic} infection

Infectivity of an infectious individual varies over time. If an individual \(j\)
gets
infected at time \(t_0^j\), then for \(t\in [t_0^j,t_0^j+T_I]\), note \(h_j
(t)=h_0
(t-t_0^j)\) its
infectivity, where \(h_0\) is the common infectivity profile of all individuals.

Within the household, individuals are mixing completely. Hence, at time \(t\), a
susceptible individual gets exposed to all infectious individuals. The 
collective force of infection is then 
\begin{equation}
	H(t) = \sum_{j\in I_t} h_j(t),
\end{equation}
and a susceptible individual gets infected with probability
\begin{equation}
	1-Q_h(t) = 1- \exp(-\beta H(t)).
\end{equation}
The term \(Q_h(t)\) is sometimes dubbed \emph{household escape probability}. 
Note that when
\(h(t)=h\) is constant, \(H(t)=hi_t\), and the escape
probability can be written as \(Q(t) = \exp(-\beta)^{i_t}\), which is the case 
of the classical Reed-Frost model.

In addition to infections within the household, each susceptible individual can
also be infected in the community. If \(A\) is the probability of that event,
which is assumed to be independent from within-household infection, then total
infection probability at time \(t\) is then

\[
	P(t)=1-Q(t) = 1-A\exp(-\beta H(t)).
\]

This process of infections happens at each timestep, independently for each
susceptible individual, so that at time \(t+1\), the number of new infections is
binomially distributed with parameters \(s_t\) and \(P(t)\). The ensuing process
\((s_t,i_t)\) is not Markovian (because of the fixed recovery time \(T_I\)).

\section*{Observed data}

We assume that the data contain the number of individuals in each household,
along with symptom onset times of symptomatic individuals. Additional covariates
can be integrated in the model (see below, \emph{Possible extensions}).

\section*{Parameters}

\begin{itemize}
\item Duration of infection \(T_I=T_p+T_s\)
\item Infectivity profile \(h_0(t),\ t\in [0,T_I]\)
\item Transmission parameter \(\beta >0\)
\item Community infection probability \(A\)
\item Rate of asymptomatic infection \(F\)
\end{itemize}

\section*{Possible extensions}

There are a number of important aspects of Covid-19 infection that we would like
to address with these data. This requires refining the model to take individual
covariates into account, such as age (which should be modelled as a categorical
variable), severity of disease or the frequency of outside contacts. 

For now, we assume that infectious individuals have the same transmission
characteristics, but this could very well not be the case. An additional
parameter \(v_j\) could be introduced to represent heterogeneity in
transmissibility, such that \(h_j(t) = v_j\times h_0(t-t_0^j)\). The
individual-level parameters  \(v_j\) could be drawn from a common distribution,
such as a Gamma distribution, which allows for varying levels of heterogeneity 
(see Fraser et al. 2011). The same could also be done for the infectious period 
\(T_I\). 

The impact of asymptomatic infections should also be modelled carefully, since
it is not known at this time whether they can be infectious or not. Following 
(Fraser et al. 2011), we could distinguish between uninfectious and infectious
asymptomatic disease.

A very important factor is the presence of non-Covid-19 infections in the
community, which can lead to symptoms and outbreaks similar to Covid-19. 

\end{document}