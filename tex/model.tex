\documentclass[a4paper]{article}

\usepackage[utf8]{inputenc}
\usepackage{amsmath}

\title{A Reed-Frost household model for Covid-19}

\begin{document}

\maketitle

\section*{Model description}

We consider a finite population divided into \emph{households}, with individuals
interacting both within and outside of their respective households. We will 
focus on modeling the spread within the household, while taking the possibility
of community-acquired infection into account. 

For a given household with \(m\) individuals, consider at each time \(t\) the
set \(S_t\) of susceptible individuals and the set \(I_t\) of infectious
individuals. The number of such individuals will be noted \(s_t\) and \(i_t\)
respectively. Given the long incubation time of Covid-19 and the possibility of
presymptomatic transmission, we will also consider the set \(E_t\) of 
\emph{exposed} individuals, who have yet to develop symptoms. Exposed 
individuals then become infectious with probability \(p_I\) at each timestep, so
that the total length of the latent period is geometric with parameter \(p_I\).

The level of infectiousness of an infectious individual \(j\) will be noted 
\(h_j(t)\). We will assume that infectiousness decreases with time in a 
geometric fashion, such that, if \(j\) is infectious, 
\begin{equation}
	h_j(t+1) = \gamma h_j(t).
\end{equation}
Within the household, individuals are mixing completely. Hence, at time \(t\), a
susceptible individual gets exposed to all infectious individuals. The 
collective force of infection is then 
\begin{equation}
	H_t = \sum_{j\in I_t} h_j(t),
\end{equation}
and a susceptible individual becomes exposed (is infected with the virus) with
probability
\begin{equation}
	1-Q_h(t) = 1- \exp(-\beta H_t).
\end{equation}
The term \(Q_h(t)\) is sometimes dubbed \emph{household escape probability}. 
Note that when \(h(t)=h_0\) is constant, \(H_t=h_0 i_t\), and the escape
probability can be written as \(Q(t) = \exp(-\beta h_0)^{i_t}\), which is the
case of the classical Reed-Frost model.

In addition to infections within the household, each susceptible individual can
also be infected in the community. If \(1-A\) is the probability of that event,
which is assumed to be independent from within-household infection, then total
infection probability at time \(t\) is 

\[
	P(t)=1-Q(t) = 1-A\exp(-\beta H_t).
\]

This process of infections happens at each timestep, independently for each
susceptible individual, so that at time \(t+1\), the number of new infections is
binomially distributed with parameters \(s_t\) and \(P(t)\). We will study the
Markovian process
\begin{equation}
	X_t=(s_t,e_t,\Delta i_t,H_t),
\end{equation}
whose transitions can be summed up as follow:

\begin{align}
	e_{t+1} & \sim \text{Bin}(s_t,P(t)), \\
	\Delta i_{t+1} & \sim \text{Bin}(e_t,p_I), \\
	H_{t+1} & = \gamma H_t + h_0 \Delta i_{t+1}, \\
	s_{t+1} & = s_t - \Delta i_{t+1}.
\end{align}

\section*{Observation model}

We assume that the data contain the number \(s_0\) of individuals in each
household, along with symptom onset times of symptomatic individuals. In our
context, we will assume that infectious individuals are symptomatic with
probability \(F\) and we will write \(Y_t \sim \text{Bin}(\Delta i_t,F)\) for 
the observed number of newly-infected, symptomatic individuals. 

This, and the Markov property of the hidden process \(X\) makes inference 
through classical HMM algorithms possible, such as particle MCMC, or iterated
filtering. As a sanity check, it should also be possible to fit the classical
Reed-Frost model to the final size data \((s_0,i_\infty)\) as in 
\cite{Cauchemez2014} to
estimate some of the same parameters as in this model.

\section*{Parameters}

\begin{itemize}
\item Rate of decrease of infectiousness \(\gamma >0\).
\item Initial infectiousness \(h_0 >0\).
\item Transmission parameter \(\beta >0\).
\item Community escape probability \(A\).
\item Rate of asymptomatic infection \(F\).
\item Probability of an exposed individual becoming infectious per time unit 
\(p_I\).
\end{itemize}

\section*{Possible extensions}

There are a number of important aspects of Covid-19 infection that we would like
to address with these data. This requires refining the model to take individual
covariates into account, such as age (which should be modelled as a categorical
variable), severity of disease or the frequency of outside contacts. 

For now, we assume that infectious individuals have the same transmission
characteristics, but this could very well not be the case. The initial
infectiousness \(h_0\) could be drawn from a common distribution,
such as a Gamma distribution, which allows for varying levels of heterogeneity 
(see \cite{Fraser2011}). Also, we did not include the age structure of
households, which will have to be done at some point since transmissibility
between age groups is a crucial knowledge gap at this time.

The impact of asymptomatic infections should also be modelled carefully, since
it is not known at this time whether they can be infectious or not. Following 
(\cite{Fraser2011}), we could distinguish between uninfectious and infectious
asymptomatic disease.

A very important factor is the presence of non-Covid-19 infections in the
community, which can lead to symptoms and outbreaks similar to Covid-19. This
represents a major bias in the data. This can be resolved in several ways: the
simplest one is to independently estimate the proportion of influenza-like
illnesses who are in fact due to Covid-19 disease (for instance, by using the
rate of positive PCR tests as a proxy), and adjust the parameters accordingly. A
more sophisticated approach would be to finely distinguish between symptoms,
according to their predictive value for Covid-19.

For now, we did not include the possibility of presymptomatic transmission, for
simplicity's sake. In order to take this into account, we could add a second
Exposed compartment for asymptomatic and infectious individuals. 

\newcommand{\etalchar}[1]{$^{#1}$}
\begin{thebibliography}{FCK{\etalchar{+}}11}

\bibitem[CFF{\etalchar{+}}14]{Cauchemez2014}
Simon Cauchemez, Neil~M. Ferguson, Annette Fox, Le~Quynh Mai, Le~Thi Thanh,
  Pham~Quang Thai, Dang~Dinh Thoang, Tran~Nhu Duong, Le~Nguyen {Minh Hoa},
  Nguyen {Tran Hien}, and Peter Horby.
\newblock {Determinants of Influenza Transmission in South East Asia: Insights
  from a Household Cohort Study in Vietnam}.
\newblock {\em PLoS Pathogens}, 10(8):2--9, 2014.

\bibitem[FCK{\etalchar{+}}11]{Fraser2011}
Christophe Fraser, Derek~A.T. Cummings, Don Klinkenberg, Donald~S. Burke, and
  Neil~M. Ferguson.
\newblock {Influenza transmission in households during the 1918 pandemic}.
\newblock {\em American Journal of Epidemiology}, 174(5):505--514, 2011.

\end{thebibliography}


\end{document}